% ****** Start of file aipsamp.tex ******
%
%   This file is part of the AIP files in the AIP distribution for REVTeX 4.
%   Version 4.1 of REVTeX, October 2009
%
%   Copyright (c) 2009 American Institute of Physics.
%
%   See the AIP README file for restrictions and more information.
%
% TeX'ing this file requires that you have AMS-LaTeX 2.0 installed
% as well as the rest of the prerequisites for REVTeX 4.1
%
% It also requires running BibTeX. The commands are as follows:
%
%  1)  latex  aipsamp
%  2)  bibtex aipsamp
%  3)  latex  aipsamp
%  4)  latex  aipsamp
%
% Use this file as a source of example code for your aip document.
% Use the file aiptemplate.tex as a template for your document.
\documentclass[%
 aip,
 jmp,
 amsmath,
 amssymb,
%preprint,%
 reprint,%
%author-year,%
%author-numerical,%
 numerical,
 longbibliography,
]{revtex4-1}

\usepackage{graphicx}% Include figure files
\graphicspath{{images/}}
\usepackage{dcolumn}% Align table columns on decimal point
\usepackage{bm}% bold math
\usepackage{url}
\usepackage{float}
\usepackage{silence}
\usepackage{tabularx}
\usepackage{verbatimbox}
\WarningFilter{revtex4-1}{Repair the float}
%\usepackage[mathlines]{lineno}% Enable numbering of text and display math
%\linenumbers\relax % Commence numbering lines

\begin{document}

%\preprint{AIP/123-QED}

\title[Laboratory 2]{Introduction to AC Measurements} % Force line breaks with \\

\author{Kevin "Yama" Keyser}
 \email{kk8r8@mail.umck.edu}
\affiliation{ 
	University of Missouri-Kansas City
	%\\This line break forced with \textbackslash\textbackslash
}%

%\date{\today}% It is always \today, today,
             %  but any date may be explicitly specified

\begin{abstract}
In Laboratory 2, we expand our understanding of electronic circuits from Laboratory 1.
Instead of DC measurements, we are now working with AC measurements. From there, we 
will cover concepts of frequency, frequency dependence, and the concepts behind filters, 
specifically high pass and low pass filters. We will also get an introduction to Bode
plots.
\end{abstract}

%\keywords{Operational Amplifier}%Use showkeys class option if keyword
                              %display desired
\maketitle

%\begin{quotation}
%The ``lead paragraph'' is encapsulated with the \LaTeX\ 
%\verb+quotation+ environment and is formatted as a single paragraph before the first section heading. 
%(The \verb+quotation+ environment reverts to its usual meaning after the first sectioning command.) 
%Note that numbered references are allowed in the lead paragraph.
%
%The lead paragraph will only be found in an article being prepared for the journal \textit{Chaos}.
%\end{quotation}

\section{Background}

There isn't much background needed for this lab. One of the main factors is making sure
that we have a solid understanding of how BNC connectors should be used, and the
ins-and-outs of how to work an oscilloscope. 

\section{Procedure}

The procedures for Laboratory 2 is very straight forward, so the Laboratory 2 sheet will be attached at the end of this write up.

\section{Presentation of Data}

	\subsection{2-2: AC Voltage Measurements}
	
		\begin{tabularx}{0.45\textwidth}[t]{| X | X | X | X |}
		\hline
		\multicolumn{4}{|c|}{Sine Wave (DC Values)}\\
		\multicolumn{4}{|c|}{5Vpp, 0 Offset}\\		
		\hline
			\multicolumn{1}{|c|}{Freq (Hz)} & 
			\multicolumn{1}{c|}{DC Min} & 
			\multicolumn{1}{c|}{DC Max} &
			\multicolumn{1}{c|}{DC Avg} \\ 
		\hline
		10 & -0.68 & 0.697 & 0\\ \hline
		20 & -0.109 & 0.121 & 0.007\\ \hline
		50 & 0.006 & 0.006 & 0.006\\ \hline
		1k & 0.006 & 0.006 & 0.006\\ \hline
		2k & 0.007 & 0.006 & 0.007\\ \hline
		5k & 0.007 & 0.007 & 0.007\\ \hline
		10k & 0.007 & 0.007 & 0.007\\ \hline
		20k & 0.007 & 0.007 & 0.007\\ \hline
		50k & 0.007 & 0.007 & 0.007\\ \hline
		\end{tabularx}
	
		\begin{tabularx}{0.45\textwidth}[t]{| X | X | X | X |}
		\hline
		\multicolumn{4}{|c|}{Sine Wave (AC Values)}\\
		\multicolumn{4}{|c|}{5Vpp, 0 Offset}\\
		\hline
			\multicolumn{1}{|c|}{Freq (Hz)} & 
			\multicolumn{1}{c|}{DC Min} & 
			\multicolumn{1}{c|}{DC Max} &
			\multicolumn{1}{c|}{DC Avg} \\ 
		\hline
		10  & 1.702 & 1.736 & 1.719\\ \hline
		20  & 1.752 & 1.752 & 1.752\\ \hline
		50  & 1.763 & 1.763 & 1.763\\ \hline
		1k  & 1.763 & 1.763 & 1.766\\ \hline
		2k  & 1.756 & 1.756 & 1.756\\ \hline
		5k  & 1.555 & 1.555 & 1.555\\ \hline
		10k & 1.104 & 1.113 & 1.108\\ \hline
		20k & 0.657 & 0.682 & 0.672\\ \hline
		50k & 0.353 & 0.420 & 0.392\\ \hline
		\end{tabularx}
	
		\begin{tabularx}{0.45\textwidth}[t]{| X | X | X | X |}
		\hline
		\multicolumn{4}{|c|}{Square Wave (DC Values)}\\
		\multicolumn{4}{|c|}{5Vpp, 0 Offset}\\		
		\hline
			\multicolumn{1}{|c|}{Freq (Hz)} & 
			\multicolumn{1}{c|}{DC Min} & 
			\multicolumn{1}{c|}{DC Max} &
			\multicolumn{1}{c|}{DC Avg} \\ 
		\hline
		10 & -0.869 & 0.898 & -0.031\\ \hline
		20 & -0.131 & 0.161 & 0.015 \\ \hline
		50 & 0.015 & 0.015 & 0.015 \\ \hline
		1k & 0.015 & 0.015 & 0.015 \\ \hline
		2k & 0.015 & 0.015 & 0.015 \\ \hline
		5k & 0.015 & 0.015 & 0.015 \\ \hline
		10k & 0.015 & 0.015 & 0.015 \\ \hline
		20k & 0.015 & 0.015 & 0.015 \\ \hline
		50k & 0.015 & 0.015 & 0.015 \\ \hline
		\end{tabularx}
	
		\begin{tabularx}{0.45\textwidth}[t]{| X | X | X | X |}
		\hline
		\multicolumn{4}{|c|}{Square Wave (AC Values)}\\
		\multicolumn{4}{|c|}{5Vpp, 0 Offset}\\
		\hline
			\multicolumn{1}{|c|}{Freq (Hz)} & 
			\multicolumn{1}{c|}{DC Min} & 
			\multicolumn{1}{c|}{DC Max} &
			\multicolumn{1}{c|}{DC Avg} \\ 
		\hline
		10  & 2.455 & 2.469 & 2.463\\ \hline
		20  & 2.480 & 2.480 & 2.480\\ \hline
		50  & 2.482 & 2.482 & 2.482\\ \hline
		1k  & 2.411 & 2.418 & 2.415\\ \hline
		2k  & 2.334 & 2.348 & 2.341\\ \hline
		5k  & 2.005 & 2.027 & 2.016\\ \hline
		10k & 1.418 & 1.456 & 1.436\\ \hline
		20k & 0.869 & 0.913 & 0.889\\ \hline
		50k & 0.515 & 0.592 & 0.566\\ \hline
		\end{tabularx}
	
		\begin{tabularx}{0.45\textwidth}[t]{| X | X | X | X |}
		\hline
		\multicolumn{4}{|c|}{Triangle Wave (DC Values)}\\
		\multicolumn{4}{|c|}{5Vpp, 0 Offset}\\		
		\hline
			\multicolumn{1}{|c|}{Freq (Hz)} & 
			\multicolumn{1}{c|}{DC Min} & 
			\multicolumn{1}{c|}{DC Max} &
			\multicolumn{1}{c|}{DC Avg} \\ 
		\hline
		10 & -0.555 & 0.567 & 0.041 \\ \hline
		20 & -0.087 & 0.100 & 0.004 \\ \hline
		50 & 0.007 & 0.007 & 0.007 \\ \hline
		1k & 0.007 & 0.007 & 0.007 \\ \hline
		2k & 0.007 & 0.007 & 0.007 \\ \hline
		5k & 0.007 & 0.007 & 0.007 \\ \hline
		10k & 0.007 & 0.007 & 0.007 \\ \hline
		20k & 0.007 & 0.007 & 0.007 \\ \hline
		50k & 0.007 & 0.007 & 0.007 \\ \hline
		\end{tabularx}
	
	\begin{tabularx}{0.45\textwidth}[t]{| X | X | X | X |}
	\hline
	\multicolumn{4}{|c|}{Triangle Wave (AC Values)}\\
	\multicolumn{4}{|c|}{5Vpp, 0 Offset}\\
	\hline
		\multicolumn{1}{|c|}{Freq (Hz)} & 
		\multicolumn{1}{c|}{DC Min} & 
		\multicolumn{1}{c|}{DC Max} &
		\multicolumn{1}{c|}{DC Avg} \\ 
	\hline
	10  & 1.380 & 1.413 & 1.396\\ \hline		
	20  & 1.430 & 1.430 & 1.430\\ \hline
	50  & 1.442 & 1.442 & 1.442\\ \hline
	1k  & 1.444 & 1.444 & 1.444\\ \hline
	2k  & 1.433 & 1.433 & 1.433\\ \hline
	5k  & 1.264 & 1.267 & 1.265\\ \hline
	10k & 0.894 & 0.902 & 0.900\\ \hline
	20k & 0.532 & 0.556 & 0.545\\ \hline
	50k & 0.282 & 0.342 & 0.312\\ \hline
	\end{tabularx}
	
	\subsection{2-3: Time/Frequency Measurements}
	
	No tabular data.
	
	\subsection{2-4: Low-Pass and High-Pass Filters}
	
	\begin{tabularx}{0.45\textwidth}[t]{| X | X | X |}
	\hline
	\multicolumn{3}{|c|}{Low-Pass Filter}\\
	\hline
		\multicolumn{1}{|c|}{Freq (Hz)} & 
		\multicolumn{1}{c|}{$V_{in}$} & 
		\multicolumn{1}{c|}{$V_{out}$} \\ 
	\hline
	10   & 8.160 & 8.680 \\ \hline
	20   & 9.520 & 8.600 \\ \hline
	50   & 10.08 & 8.560 \\ \hline
	100  & 10.08 & 7.440 \\ \hline
	200  & 10.08 & 5.320 \\ \hline
	500  & 10.08 & 2.640 \\ \hline
	1k   & 10.08 & 1.360 \\ \hline
	2k   & 10.08 & 0 \\ \hline
	5k   & 10.08 & 0 \\ \hline
	10k  & 10.08 & 0 \\ \hline
	20k  & 10.08 & 0 \\ \hline
	50k  & 10.08 & 0 \\ \hline
	100k & 10.08 & 0 \\ \hline	
	\end{tabularx}
	
	\begin{tabularx}{0.45\textwidth}[t]{| X | X | X |}
	\hline
	\multicolumn{3}{|c|}{High-Pass Filter}\\
	\hline
		\multicolumn{1}{|c|}{Freq (Hz)} & 
		\multicolumn{1}{c|}{$V_{in}$} & 
		\multicolumn{1}{c|}{$V_{out}$} \\ 
	\hline
	10   & 8.160 & 0 \\ \hline
	20   & 9.520 & 1.280 \\ \hline
	50   & 10.08 & 3.200 \\ \hline
	100  & 10.08 & 5.520 \\ \hline
	200  & 10.08 & 7.840 \\ \hline
	500  & 10.08 & 9.400 \\ \hline
	1k   & 10.08 & 9.840 \\ \hline
	2k   & 10.08 & 10.00 \\ \hline
	5k   & 10.08 & 10.00 \\ \hline
	10k  & 10.08 & 10.00 \\ \hline
	20k  & 10.08 & 10.00 \\ \hline
	50k  & 10.08 & 10.00 \\ \hline
	100k & 10.08 & 10.00 \\ \hline	
	\end{tabularx}
	
\section{Discussion} \label{Section:Discussion}
	
	\subsection{2-2: AC Voltage Measurements}
	
	The AC and DC voltage measurements with the DMM accomplished a couple of things.
	For an AC input, the DMM will read the average value for DC. It has to sample
	in order to get these readings, and it averages over the samples. When you have
	a voltage that is a function of frequency, your "`average"' value will always be
	what your offset is. This is why the DC readings on our tables, regardless of the
	wave, came out to nearly zero.
	
	For the AC readings, we found out the limitations of the DMM versus the multimeter.
	This makes sense when you consider what a DMM is likely to be used for, which is mostly
	consumer electronics. This means you'll have the highest efficiencies at 50-60Hz. Anything
	below, and you'll get weird readings. Anything higher and you'll get eventually get readings
	of zero. How quickly you go to zero, and what your readings are for the AC mode on the DMM
	will depend on your wave.
	
	This goes to show that for basic uses, like consumer electronics, a DMM will most likely
	suit all your needs. For anything more complex, it may require an oscilloscope.
	
	\subsection{2-3: Time/Frequency Measurements}
	
	Though the DMM did a great job of giving roughly the same measurements as the oscilloscope
	for the frequency/time measurements, what it lacks is granularity. Once again, looking back
	to our last experiment, this is due to the uses of the DMM versus the uses of an 
	oscilloscope. Where the oscilloscope gave us a reading of 20.04ms, the DMM gave us a reading
	of 0.02 seconds. Though they're roughly the same value, the oscilloscope gives us access
	to more granular data, which can come in handy for sensitive experiments.
	
	\subsection{2-4: Low-Pass and High-Pass Filters}
	
	The High-Pass and Low-Pass filter experiments show us how we can block certain frequencies
	depending on the output voltage. After looking it up online, we can find that the industry
	standard for the cutoff point for these filters is in the form of 
	$20*log(\frac{V_{out}}{V_{in}}) \geq -3dB$. Anything below -3dB is filtered out.
	
	For the Low-Pass filter, the formula for $V_{out}$ is 
	\[ V_{out} = V_{in}*\frac{X_c}{\sqrt{R^2+X_c^2}} \]
	Where R is resistance of the circuit, and $X_c$ is the capacitive reactance, given by
	$X_c = -\frac{1}{\omega C} = -\frac{1}{2\pi fC}$
	
	For the High-Pass filter, the formula for $V_{out}$ is
	\[ V_{out} = V_{in}*\frac{R}{\sqrt{R^2+X_c^2}} \]
	
	You can see that for the Low-Pass filter, as the frequency goes to infinity, $X_c$ goes
	to zero, making $V_{out}$ go to zero, and the opposite is true for the High-Pass filter.
	
	For filtering out specific frequencies, or having specific cut offs, you choose your
	frequency first, and set choose your resistor and capacitor that will get you that -3dB
	cutoff point.

\section{Conclusion}

Though it took a little digging ti figure out why the DMM would be less sensitive (other than
the obvious answer that it's cheaper), we learned that oscilloscopes are meant for use in all
situations, where most DMMs are built for consumer electronics. This means they are usually
less sensitive, which is why we normally opt for the oscilloscope for our experiments. 

Beyond that, we learned the formulas for $V_{out}$ for both High-Pass and Low-Pass filters,
and how to mathematically formulate our own filters for our own needs. We can also expand this
knowledge, and take the output of our High-Pass or Low-Pass filter, and feed it into the other
filter so we can filter down to a small window, or band, so we only allow a limited number of
frequencies in.

\end{document}